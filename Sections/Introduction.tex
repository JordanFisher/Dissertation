Peristalsis is the predominant mechanism of action in a plethora of biological phenomena, from earthworm mobility to gastrointestinal and esophageal transport. Peristalsis is utilized in many mechanical fluid pumps, often because of their ability to effectively transport highly viscous fluids. In both biological and mechanical systems,  the fluid internal to the pump may be non-Newtonian. Such is the case for peristalsis in the oviduct and uterus where the transported biological fluid is highly complex.

Recent work has begun the investigation of peristalsis of an Oldroyd B (OB) fluid~\cite{teran2008peristaltic, chrispell2010peristaltic}. 
A marked difference has been reported with respect to the Newtonian flow counterpart, particularly the non-linear relationship between mean flow through the pump and the Weissenberg number of the fluid. It has been noted in these works that the polymeric stress of the transported fluid can lead to very strong normal stresses at the interface with the pump. These stresses present a difficult numerical problem. In order to correctly model the structure of the pump wall very stiff forces must be employed, leading to very small time-step constraints. Such numerical constraints (stiffness) are particularly pernicious for explicit Immersed Boundary (IB) methods, as used in~\cite{teran2008peristaltic, chrispell2010peristaltic}. There,  the numerical stiffness  limited the range of parameters ameanable to investigate. In Section~\ref{Sec:Methodology},  we present a new numerical method coupling a viscoelastic fluid solver to a novel, highly efficient semi-implicit implementation of the IB method, presented in more detail in~\cite{IBM_Implicit2D, IBM_Implicit3D}. The implicit methodology allows us to explore extreme parameter regimes previously out of reach of numerical algorithms. In particular, when the waves of peristalsis have very large amplitude, nearly occluding the channel, the normal stresses on the wall become extremely large. In addition, for very large Weissenberg numbers the transported fluid can develop very strong stresses, even for moderate occlusions. The implicit methodology allows us to robustly probe both of these regimes. Previous work has hinted at interesting behavior as the pump occlusion is increased, including substantial effects on the pump's rate of flow. We probe these effects  for substantially larger occlusions than previously possible. In Section~\ref{sec:Results} we detail new behavior observed, including evidence for a finite-time blow up of the OB equations.

